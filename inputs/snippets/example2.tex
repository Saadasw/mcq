

\documentclass[12pt]{article}
\usepackage{amsmath, amssymb}
\usepackage{polyglossia}
\usepackage{geometry}
\usepackage{enumitem}
\geometry{a4paper, margin=1in}
\setmainlanguage{bengali}
\newfontfamily\bengalifont[Script=Bengali]{Noto Sans Bengali}

% Custom formatting
\setlength{\parindent}{0pt}
\setlength{\parskip}{0.3em}

% Custom enumerate for Bengali letters
\newlist{benglienum}{enumerate}{1}
\setlist[benglienum]{label=(\alph*), leftmargin=2em}

\begin{document}

\begin{center}
{\Large \textbf{প্রথম অধ্যায়: ম্যাট্রিক্স ও নির্ণায়ক}}\\
\vspace{0.3cm}
{\large \textbf{★★ ম্যাট্রিক্স ও ম্যাট্রিক্সের প্রকারভেদ}}
\end{center}

\vspace{0.5cm}

\begin{enumerate}

% Question 1
\item নিচের কোনটি $3 \times 2$ ক্রমের ম্যাট্রিক্স? \textbf{(সহজ)}
\begin{benglienum}
    \item $\begin{pmatrix} x & y & z \\ 1 & 2 & 3 \end{pmatrix}$
    \item $\begin{pmatrix} 0 & 5 \\ 6 & 0 \\ a & b & c \end{pmatrix}$
    \item $(1 \quad 2 \quad 3)$a
    \item $\begin{pmatrix} d & e & f \\ g & h \end{pmatrix}$ \checkmark
\end{benglienum}


\end{enumerate}

\end{document}